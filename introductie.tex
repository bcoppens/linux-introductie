\documentclass[a4paper]{memoir}
\usepackage[dutch]{babel}

\epigraphfontsize{\small\itshape}
\setlength\epigraphwidth{8cm}
\setlength\epigraphrule{0pt}

\begin{document}

\epigraphfontsize{\small\itshape}

\chapter{Inleiding tot Linux}
\epigraph{``It’s a UNIX system, I know this!''}{--- \textup{Lex Murphy}, Jurassic Park}

De komende drie weken gaan we een korte inleiding geven tot Linux. Maar wat is Linux? Strikt genomen is Linux
niet meer dan een besturingssysteemkern (\emph{operating systen kernel} die er achter de schermen voor zorgt dat je applicaties kan starten, dat je
er iets gebeurt als je op je toetsenbord typt, etc. Het zijn echter de applicaties die er op draaien die een interface voorzien waar je als gebruiker mee interageert.
Standaard zal je merken dat die interface ook gewoon grafisch is, net zoals je gewend bent bij Windows of OSX. Er zijn zelfs meerdere verschillende grafische interfaces
beschikbaar die vaak gebruikt worden onder Linux; de meest voorkomende zijn KDE en Gnome.

Die grafisch intu"itieve interfaces zijn echter niet waarover deze sessies zullen gaan, die wijzen zichzelf immers uit. Deze labo's gaan over
de command line interface, die helaas (veel) minder intu"itief is, maar daarom niet minder belangrijk is om wat ervaring mee te hebben. Met enige kennis van de command
line applicaties, en hoe je die kan combineren met elkaar, kan je je leven later (hopelijk) veel makkelijker maken (en dat niet alleen in de practica van \emph{computerarchitectuur},
maar ook in andere vakken van de opleidingen computerwetenschappen en elektrotechniek, en misschien zelfs je thesis).

\section{Organizatie van de practica}




\end{document}
